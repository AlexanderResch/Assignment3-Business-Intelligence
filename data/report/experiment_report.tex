\documentclass[sigconf]{acmart}

\AtBeginDocument{ \providecommand\BibTeX{ Bib\TeX } }
\setcopyright{acmlicensed}
\copyrightyear{2025}
\acmYear{2025}
\acmDOI{XXXXXXX.XXXXXXX}

\acmConference[BI 2025]{Business Intelligence}{-}{-}

\begin{document}

\title{BI2025 Experiment Report - Group 54}
%% ---Authors: Dynamically added ---

          \author{Alexander Resch}
          \authornote{Student A, Matr.Nr.: 12017130}
          \affiliation{
            \institution{TU Wien}
            \country{Austria}
          }
          
          \author{Jakob Kimeswenger}
          \authornote{Student B, Matr.Nr.: 12122531}
          \affiliation{
            \institution{TU Wien}
            \country{Austria}
          }
          

\begin{abstract}
  This report documents the machine learning experiment for Group 54, following the CRISP-DM process model.
\end{abstract}

\ccsdesc[500]{Computing methodologies~Machine learning}
\keywords{CRISP-DM, Provenance, Knowledge Graph, Machine Learning}

\maketitle

%% --- 1. Business Understanding ---
\section{Business Understanding}

\subsection{Data Source and Scenario}
We decided to use the "Corporate Credit Rating with Financial Ratios" dataset available on Kaggle.
The dataset contains a set of financial ratios measuring liquidity, leverage and profitability for multiple companies, together with a corporate credit rating label.
The scenario is a financial institution that wants to assess corporate credit risk based on financial statement information.
The model should assist analysts in assigning credit ratings in decision on creditworthiness.

\subsection{Business Objectives}
-) Support analysts with consistent credit rating decisions for coporate clients.
-) Identify companies with high credit risk early, to reduce credit losses.
-) Decrease the effort for manual review for low risk companies.
-) At new credit applications, speed up the rating decisions.

%% --- 2. Data Understanding ---
\section{Data Understanding}
\textbf{Dataset Description:} Credit ratings and financial ratios for corporations.

The following features were identified in the dataset:

\begin{table}[h]
  \caption{Raw Data Features}
  \label{tab:features}
  \begin{tabular}{lp{0.2\linewidth}p{0.4\linewidth}}
    \toprule
    \textbf{Feature Name} & \textbf{Data Type} & \textbf{Description} \\
    \midrule
    binary\_rating & integer> & Good vs bad rating indicator \\
    current\_ratio & double> & Current assets divided by current liabilities \\
    debt\_equity\_ratio & double> & Leverage ratio total debt relative to equity \\
    rating & string> & Long term credit rating symbol (for example AAA, BBB-) \\
    rating\_date & dateTime> & Date at which the credit rating was assigned \\
    \bottomrule
  \end{tabular}
\end{table}

%% --- 3. Data Preparation ---
\section{Data Preparation}
\subsection{Data Cleaning}
Describe your Data preparation steps here and include respective graph data.


%% --- 4. Modeling ---
\section{Modeling}

\subsection{Hyperparameter Configuration}
The model was trained using the following hyperparameter settings:

\begin{table}[h]
  \caption{Hyperparameter Settings}
  \label{tab:hyperparams}
  \begin{tabular}{lp{0.4\linewidth}l}
    \toprule
    \textbf{Parameter} & \textbf{Description} & \textbf{Value} \\
    \midrule
    Learning Rate & ... & 1.23 \\
    \bottomrule
  \end{tabular}
\end{table}

\subsection{Training Run}
A training run was executed with the following characteristics:
\begin{itemize}
    \item \textbf{Algorithm:} Random Forest Algorithm
    \item \textbf{Start Time:} 2025-12-16 17:24:54
    \item \textbf{End Time:} 2025-12-16 17:24:54
    \item \textbf{Result:} R-squared Score = 1.2300
\end{itemize}

%% --- 5. Evaluation ---
\section{Evaluation}

%% --- 6. Deployment ---
\section{Deployment}

\section{Conclusion}

\end{document}
